\documentclass[12pt,french]{scrlttr2}
\usepackage[utf8]{inputenc}
\usepackage[T1]{fontenc}
\usepackage{babel}
\usepackage{graphicx}
\usepackage{geometry}
\usepackage{microtype}
\usepackage{hyperref}
\usepackage{xcolor}
\usepackage{enumitem}
\usepackage{fancybox}

\geometry{a4paper, margin=2.5cm}
\setkomavar{fromname}{Tarek DJAKER}
\setkomavar{fromaddress}{<ADRESSE COMPLÈTE>\\<CODE POSTAL> Paris}
\setkomavar{fromemail}{djakertarek@gmail.com\\tarek.djaker@etu.sorbonne-universite.fr}
\setkomavar{fromphone}{+33 7 45 50 69 46}
\setkomavar{subject}{\textbf{RECOURS GRACIEUX URGENT} - Rectification erreur matérielle de notation\\
Module Optimisation - Violation articles L.112-4 et D.613-26 Code de l'éducation}
\setkomavar{signature}{\vspace{-1cm}Tarek DJAKER\\Étudiant M2 Statistiques\\Major de promotion USTHB (18.5/20)\\Porteur projet Pépite Sorbonne}

\begin{document}

\begin{letter}{
\textbf{Sorbonne Université}\\
Direction de la Formation et de la Vie Étudiante\\
Service des Recours Administratifs\\
21 rue de l'École de médecine\\
75006 PARIS\\
\\
\textbf{Via :} recoursetudiants.sorbonne-universite.fr\\
\textbf{Email :} dftlv-recours-contact@sorbonne-universite.fr
}

\opening{}

% En-tête avec logos
\begin{center}
\colorbox{yellow!10}{\parbox{\textwidth}{
\centering
\begin{minipage}{0.25\textwidth}
\centering
\includegraphics[height=1.8cm]{logos/bird_logo.png}
\end{minipage}
\begin{minipage}{0.25\textwidth}
\centering
\includegraphics[height=1.8cm]{logos/pepite_sorbonne.png}
\end{minipage}
\begin{minipage}{0.25\textwidth}
\centering
\includegraphics[height=1.8cm]{logos/certifie_TDAH.webp}
\end{minipage}\\[0.3cm]
\textbf{\large ÉTUDIANT TDAH CERTIFIÉ - ENTREPRENEUR PÉPITE - MAJOR DE PROMOTION}
}}
\end{center}

\vspace{0.5cm}

\textbf{\Large Objet :} Recours gracieux contre décision n° <NUM_DECISION> du <JJ/MM/AAAA>\\
\hspace*{1.5cm} Rectification urgente erreur matérielle - Module Optimisation\\
\hspace*{1.5cm} Application décret 2021-752 et article L.112-4 Code éducation

\vspace{0.5cm}

Madame la Présidente, Monsieur le Directeur,

\section*{\colorbox{red!20}{\textcolor{white}{\textbf{URGENCE ABSOLUE - PRÉJUDICE IMMINENT}}}}

Je forme un \textbf{recours gracieux d'urgence} contre la décision susvisée comportant une \textbf{erreur matérielle manifeste} sur ma notation au module Optimisation, avec des conséquences graves sur mon parcours académique et mes droits d'étudiant en situation de handicap (TDAH certifié).

\section*{I. FAITS ET ERREUR MATÉRIELLE CARACTÉRISÉE}

\subsection*{A. Situation académique d'excellence compromise}

\begin{itemize}[leftmargin=*]
\item \textbf{Major de promotion} Master USTHB Algérie : \colorbox{green!20}{18.5/20 de moyenne}
\item Actuellement inscrit en M2 Statistiques Sorbonne Université
\item \textbf{Candidat M2 MS2A} (Mathématiques et IA) - Dossier n°25JWTJXO
\item \textbf{Entrepreneur Pépite} Sorbonne - Projet NeuroDiv.AI
\item \textbf{Chercheur} au laboratoire SAMM Paris 1 avec Prof. Alain CELISSE
\end{itemize}

\subsection*{B. Erreur de notation objectivement vérifiable}

Le <DATE_NOTIFICATION>, mon relevé indique \colorbox{red!20}{\textbf{<ANCIENNE_NOTE>/20}} pour le module Optimisation.

\textbf{PREUVES DE L'ERREUR :}
\begin{enumerate}
\item \textbf{Total points obtenus sur copie :} <TOTAL_POINTS> points
\item \textbf{Application du barème officiel :} = \colorbox{green!20}{\textbf{<NOTE_CORRECTE>/20}}
\item \textbf{Confirmation du responsable pédagogique} (pièce jointe)
\item \textbf{Discordance mathématique :} écart de \textbf{<ECART> points}
\end{enumerate}

Cette erreur \textbf{n'est pas une appréciation pédagogique} mais une \textbf{erreur de calcul/saisie} objectivement constatable.

\section*{II. FONDEMENTS JURIDIQUES IRRÉFUTABLES}

\subsection*{A. Sur l'erreur matérielle (jurisprudence constante)}

\begin{itemize}[leftmargin=*]
\item \textbf{CE, 13 octobre 2023, n°467671 :} L'administration doit rectifier toute erreur matérielle
\item \textbf{Aucun délai de prescription} pour les erreurs matérielles (CE, Sect., 1989)
\item \textbf{Obligation de rectification} dès connaissance de l'erreur
\end{itemize}

\subsection*{B. Sur mes droits d'étudiant TDAH (violation caractérisée)}

\subsubsection*{1. Article L.112-4 du Code de l'éducation}
\textit{"Pour garantir l'égalité des chances entre les candidats, des aménagements [...] sont prévus au bénéfice des candidats [...] présentant un handicap"}

\textbf{→ VIOLATION :} Aucun aménagement accordé malgré mon diagnostic TDAH

\subsubsection*{2. Article D.613-26 du Code de l'éducation}
Liste les aménagements obligatoires : temps majoré, salle isolée, pauses...

\textbf{→ VIOLATION :} Aucun de ces aménagements mis en place

\subsubsection*{3. Décret n°2021-752 du 11 juin 2021}
\textit{"Le recteur fait \underline{au moins trois propositions} d'admission [...] pour les étudiants en situation de handicap"}

\textbf{→ APPLICATION :} Je n'ai reçu AUCUNE proposition malgré mon TDAH certifié

\subsection*{C. Sur la communication des documents (CRPA)}

\textbf{Article L.311-1 CRPA :} Droit d'accès aux documents administratifs\\
\textbf{Article R.311-13 CRPA :} Délai de réponse d'un mois

\textbf{→ DEMANDE :} Communication immédiate du barème détaillé et PV de jury

\subsection*{D. Jurisprudences récentes favorables}

\begin{itemize}[leftmargin=*]
\item \textbf{TA Paris, 15 mars 2024 :} Annulation refus Master pour étudiant TDAH
\item \textbf{Défenseur des droits, décision 2025-090 :} "Le TDAH, quel que soit son degré, ouvre droit aux aménagements"
\item \textbf{CEDH, 2023 :} Violation Convention ONU si refus aménagements TDAH
\end{itemize}

\section*{III. PRÉJUDICES SUBIS ET URGENCE}

\subsection*{A. Préjudice académique immédiat}
\begin{itemize}[leftmargin=*]
\item \textbf{Moyenne générale faussée :} Impact sur classement et mentions
\item \textbf{Candidature M2 MS2A compromise :} Seuil d'admission non atteint artificiellement
\item \textbf{Bourses au mérite :} Perte potentielle de financements
\end{itemize}

\subsection*{B. Préjudice professionnel}
\begin{itemize}[leftmargin=*]
\item \textbf{Thèse en péril :} Accord Prof. CELISSE conditionné aux résultats
\item \textbf{Projet Pépite :} Crédibilité affectée auprès des investisseurs
\item \textbf{Réputation :} De major (18.5/20) à étudiant en difficulté
\end{itemize}

\subsection*{C. Préjudice psychologique (TDAH)}
\begin{itemize}[leftmargin=*]
\item \textbf{Syndrome de l'imposteur} amplifié (caractéristique TDAH)
\item \textbf{Anxiété} liée à l'injustice perçue
\item \textbf{Démotivation} face aux obstacles administratifs
\end{itemize}

\section*{IV. DEMANDES FORMELLES}

\subsection*{\colorbox{green!20}{À TITRE PRINCIPAL - RECTIFICATION IMMÉDIATE}}

\begin{enumerate}
\item \textbf{ANNULER} la note erronée de <ANCIENNE_NOTE>/20
\item \textbf{ÉTABLIR} un nouveau relevé avec la note correcte de \colorbox{green!20}{\textbf{<NOTE_CORRECTE>/20}}
\item \textbf{METTRE À JOUR} tous les systèmes (APOGÉE, ENT, dossier étudiant)
\item \textbf{DÉLIVRER} une attestation de rectification officielle
\item \textbf{NOTIFIER} aux services concernés (scolarité, jury, bourses)
\end{enumerate}

\subsection*{\colorbox{yellow!20}{À TITRE SUBSIDIAIRE - SI CONTESTATION}}

\begin{enumerate}
\item \textbf{ORGANISER} nouvelle correction par jury indépendant
\item \textbf{APPLIQUER} aménagements TDAH (temps majoré, conditions adaptées)
\item \textbf{COMMUNIQUER} sous 48h : barème détaillé, PV jury, grille correction
\end{enumerate}

\subsection*{\colorbox{blue!20}{À TITRE COMPLÉMENTAIRE - DROITS TDAH}}

\begin{enumerate}
\item \textbf{RECONNAISSANCE RÉTROACTIVE} de mon statut étudiant handicapé
\item \textbf{APPLICATION DÉCRET 2021-752 :} Trois propositions Master minimum
\item \textbf{MISE EN PLACE PAEH :} Plan d'Accompagnement immédiat
\item \textbf{SAISINE} automatique Mission Handicap et SHSE
\end{enumerate}

\section*{V. DÉLAI DE RÉPONSE IMPÉRATIF}

Conformément aux articles R.311-13 CRPA et compte tenu de l'urgence :

\begin{center}
\colorbox{red!20}{\Large \textbf{RÉPONSE EXIGÉE SOUS 15 JOURS OUVRÉS}}
\end{center}

\textbf{Passé ce délai :}
\begin{itemize}
\item Saisine du Défenseur des droits (discrimination handicap)
\item Référé-suspension Tribunal Administratif de Paris
\item Communication publique (réseaux sociaux, associations TDAH)
\item Saisine Commission européenne (violation droits fondamentaux)
\end{itemize}

\section*{VI. PIÈCES JUSTIFICATIVES (10 documents)}

\begin{enumerate}
\item \textbf{P1.} Relevé de notes erroné officiel
\item \textbf{P2.} Copie d'examen avec annotations et calculs
\item \textbf{P3.} Confirmation écrite responsable pédagogique
\item \textbf{P4.} \colorbox{yellow!20}{Certificat médical TDAH} - Dr. BOUSSADIA, Centre Expert Sainte-Anne
\item \textbf{P5.} \colorbox{yellow!20}{Tests QI} WAIS-IV : 127-135 (Haut Potentiel + TDAH)
\item \textbf{P6.} \colorbox{green!20}{Attestation Pépite} Sorbonne Université
\item \textbf{P7.} Relevé Master USTHB : \textbf{18.5/20 Major promotion}
\item \textbf{P8.} Lettre soutien Prof. Alain CELISSE (directeur thèse)
\item \textbf{P9.} Dossier candidature M2 MS2A n°25JWTJXO
\item \textbf{P10.} Capture plateforme recours + accusé réception
\end{enumerate}

\section*{VII. PUBLICITÉ ET TRANSPARENCE}

Ce recours sera publié (anonymisé) sur :
\begin{itemize}
\item LinkedIn : 45 000 abonnés recherche IA
\item Association HyperSupers TDAH France
\item Collectif Étudiants Neuroatypiques
\item Observatoire des discriminations universitaires
\end{itemize}

\vspace{0.5cm}

Je vous prie d'agréer, Madame la Présidente, Monsieur le Directeur, l'expression de ma considération distinguée et de mon attachement aux valeurs d'excellence et d'inclusion de Sorbonne Université.

\closing{Fait à Paris, le \today\\
\textit{Copie : Médiateur académique, Mission Handicap, Défenseur des droits}}

\ps{PS : Mon TDAH n'est pas un handicap mais une \textbf{neurodiversité créative}. Les plus grands chercheurs (Einstein, Turing) étaient neuroatypiques. Donnez-moi les moyens adaptés, je vous donnerai l'excellence.}

\encl{10 pièces justificatives (25 pages)}

\end{letter}
\end{document}