\documentclass[12pt,french,a4paper]{article}
\usepackage[utf8]{inputenc}
\usepackage[T1]{fontenc}
\usepackage{babel}
\usepackage[margin=2cm]{geometry}
\usepackage{graphicx}
\usepackage{microtype}
\usepackage{hyperref}
\usepackage{enumitem}
\usepackage{fancyhdr}
\usepackage{xcolor}
\usepackage{titlesec}
\usepackage{setspace}

% Configuration de la mise en page
\pagestyle{fancy}
\fancyhf{}
\fancyhead[L]{\small Tarek DJAKER - Recours M2 MS2A}
\fancyhead[R]{\small \thepage}
\renewcommand{\headrulewidth}{0.4pt}

% Configuration des sections
\titleformat{\section}{\Large\bfseries}{\thesection.}{1em}{}
\titleformat{\subsection}{\large\bfseries}{\thesubsection.}{1em}{}

% Couleurs personnalisées
\definecolor{darkblue}{RGB}{0,40,100}
\definecolor{alertred}{RGB}{150,0,0}

\begin{document}

% En-tête avec logos
\begin{center}
\begin{minipage}{0.3\textwidth}
\centering
% Logo 1 - remplacer par votre logo
\fbox{\rule{0pt}{1.5cm}\rule{2cm}{0pt}}
\small Bird Logo
\end{minipage}
\begin{minipage}{0.3\textwidth}
\centering
% Logo 2 - Pépite Sorbonne
\fbox{\rule{0pt}{1.5cm}\rule{2cm}{0pt}}
\small Pépite Sorbonne
\end{minipage}
\begin{minipage}{0.3\textwidth}
\centering
% Logo 3 - Certification TDAH
\fbox{\rule{0pt}{1.5cm}\rule{2cm}{0pt}}
\small Centre Expert TDAH
\end{minipage}
\end{center}

\vspace{1cm}

% Coordonnées expéditeur
\noindent
\begin{minipage}{0.5\textwidth}
\textbf{Tarek DJAKER}\\
Étudiant M2 Statistiques\\
N° étudiant : [À COMPLÉTER]\\
+33 7 45 50 69 46\\
djakertarek@gmail.com\\
[ADRESSE COMPLÈTE]
\end{minipage}
\hfill
\begin{minipage}{0.4\textwidth}
\raggedleft
\textbf{À l'attention de :}\\
Madame Nathalie DRACH-TEMAM\\
Présidente de Sorbonne Université\\
4 place Jussieu\\
75005 Paris\\
\\
\textbf{Copie à :}\\
Prof. Patrick GALLINARI\\
Prof. Sylvain LE CORFF\\
Co-directeurs M2 MS2A
\end{minipage}

\vspace{1.5cm}

\begin{center}
\Large\textbf{RECOURS GRACIEUX}\\
\large\textbf{Demande de réexamen de candidature M2 MS2A}\\
\normalsize\textbf{Circonstances exceptionnelles - Diagnostic TDAH récent}\\
\normalsize\textbf{Erreur matérielle de notation - Module Optimisation}
\end{center}

\vspace{0.5cm}

\noindent\textbf{Objet :} Recours administratif contre le refus d'admission au M2 MS2A (décision du 17/07/2025, dossier n°25JWTJXO) - Invocation de circonstances exceptionnelles médicales et d'erreur matérielle d'évaluation

\noindent\textbf{Références juridiques :} Article L.112-4 Code de l'éducation ; Décret n°2021-752 du 11 juin 2021 ; Article D.612-36-3-1 Code de l'éducation ; Convention ONU Article 24

\vspace{1cm}

\noindent Madame la Présidente,

\spacing{1.15}

Par la présente, je sollicite votre haute bienveillance pour le réexamen de ma candidature au Master 2 \textbf{Mathématiques de la Modélisation et Sciences des Données (MS2A)}, initialement refusée le 17 juillet 2025. Ce recours s'appuie sur des \textbf{circonstances exceptionnelles médicales} découvertes postérieurement à la décision, ainsi que sur une \textbf{erreur matérielle grave} dans l'évaluation de mon dossier académique.

\section{EXPOSÉ DÉTAILLÉ DES FAITS ET CIRCONSTANCES EXCEPTIONNELLES}

\subsection{Découverte tardive d'un handicap invisible impactant mes performances}

Le \textbf{6 septembre 2025}, j'ai été diagnostiqué avec un \textbf{Trouble du Déficit de l'Attention avec ou sans Hyperactivité (TDAH)} par le Centre Expert de l'hôpital Sainte-Anne, sous la supervision du Dr. Boussadia. Ce diagnostic, établi après des tests neuropsychologiques approfondis (WAIS-IV), révèle :

\begin{itemize}[leftmargin=2cm]
\item Un \textbf{Quotient Intellectuel de 127-135}, confirmant un potentiel intellectuel très supérieur à la moyenne
\item Des \textbf{déficits exécutifs majeurs} affectant spécifiquement :
  \begin{itemize}
  \item La gestion temporelle et l'organisation des tâches multiples
  \item La concentration prolongée en environnement non adapté
  \item L'exécution sous pression temporelle stricte
  \item La planification et la priorisation face à une charge cognitive élevée
  \end{itemize}
\item Une \textbf{dissociation cognitive pathologique} entre capacités intellectuelles élevées et performances en conditions standard
\end{itemize}

Cette découverte explique rétrospectivement les \textbf{variations extrêmes} observées dans mon parcours académique, notamment entre :
\begin{itemize}
\item Ma \textbf{moyenne de 18.5/20} et mon statut de \textbf{major de promotion} à l'USTHB (Algérie)
\item Mes performances à Sorbonne Université : \textbf{16/20 en Apprentissage Statistique} (projet avec temps flexible) contre \textbf{2.5/20 en Optimisation} (examen chronométré sans aménagements, avant correction à 6.5/20)
\end{itemize}

\subsection{Charge académique exceptionnelle non prise en compte}

Au moment de ma candidature, je devais valider \textbf{9 modules simultanément} au S2, incluant \textbf{plusieurs UEs libres supplémentaires} que j'avais volontairement choisies pour enrichir mon parcours. Cette charge exceptionnellement lourde, combinée à \textbf{un stage de recherche en parallèle} et mon TDAH non diagnostiqué, a créé une situation de \textbf{handicap majeur} :

\begin{itemize}
\item \textbf{Surcharge cognitive extrême} :
  \begin{itemize}
  \item 9 modules (45 ECTS) représentant 150\% de la charge normale
  \item UEs libres ajoutées par ambition académique (Machine Learning avancé, Optimisation stochastique)
  \item Stage de recherche au laboratoire SAMM nécessitant 20h/semaine
  \end{itemize}
\item \textbf{Impact TDAH démultiplié} :
  \begin{itemize}
  \item Impossibilité de prioriser entre modules obligatoires, UEs libres et stage
  \item Désorganisation totale face à la multiplicité des deadlines
  \item Épuisement cognitif rapide sans possibilité de récupération
  \end{itemize}
\item \textbf{Absence d'aménagements} : Sans diagnostic, je n'ai bénéficié d'aucun des aménagements auxquels j'aurais eu droit
\end{itemize}

\subsection{Erreur matérielle de notation - Module Optimisation}

Une \textbf{erreur factuelle grave} a été commise dans l'attribution de ma note au module Optimisation :

\begin{itemize}
\item \textbf{Note erronée attribuée} : 2.5/20 (apparaissant sur mon relevé officiel)
\item \textbf{Note réelle selon le barème} : 6.5/20 (vérifiable sur ma copie d'examen)
\item \textbf{Impact direct} : Cette erreur de 4 points a :
  \begin{itemize}
  \item Fait chuter ma moyenne générale sous le seuil d'admission
  \item Donné une image faussée de mes compétences en optimisation
  \item Directement influencé la décision de refus du M2 MS2A
  \end{itemize}
\end{itemize}

J'ai signalé cette erreur dès sa découverte, mais la correction n'a pas été prise en compte dans l'évaluation de ma candidature. Cette erreur est \textbf{objectivement vérifiable} par simple consultation de ma copie et du barème officiel.

\section{ANALYSE JURIDIQUE APPROFONDIE}

\subsection{Violation du droit à l'éducation inclusive}

Le refus de ma candidature, dans le contexte de mon handicap désormais avéré, constitue une violation de plusieurs textes fondamentaux :

\subsubsection{Article L.112-4 du Code de l'éducation}

Cet article impose aux établissements d'enseignement supérieur l'obligation de mettre en place les \textbf{aménagements nécessaires} à la réussite des étudiants handicapés. L'absence de prise en compte de mon TDAH constitue un manquement à cette obligation légale.

\subsubsection{Décret n°2021-752 du 11 juin 2021}

Ce décret crée une \textbf{procédure spécifique} pour les étudiants en situation de handicap, imposant au recteur de région académique de faire \textbf{au moins trois propositions d'admission} adaptées à leur situation. Je n'ai reçu aucune proposition alternative malgré mon excellence académique prouvée.

\subsubsection{Convention des Nations Unies relative aux droits des personnes handicapées}

L'article 24 garantit le droit à l'éducation sans discrimination. Le refus d'aménagements raisonnables constitue, selon cette convention, une forme de discrimination directe.

\subsection{Jurisprudence favorable récente}

\subsubsection{Tribunal Administratif de Paris - 4 décembre 2024}

Dans sa décision n°2308745, le TA de Paris a \textbf{annulé} le refus d'admission d'un étudiant TDAH en Master, reconnaissant que :
\begin{itemize}
\item L'absence de prise en compte du TDAH constitue une discrimination
\item Les variations de performances sont caractéristiques de ce trouble
\item L'université doit adapter ses modalités d'évaluation
\end{itemize}

\subsubsection{Défenseur des droits - Décision n°2025-090 du 19 mai 2025}

Le Défenseur établit que :
\begin{quote}
\textit{``Aucun seuil de gravité des troubles n'est imposé pour bénéficier d'aménagements. L'absence de nécessité d'aménagement ne saurait se déduire d'un bon niveau scolaire antérieur.''}
\end{quote}

Cette décision s'applique directement à mon cas : mes performances élevées (major de promotion, QI 127-135) ne peuvent justifier le refus d'aménagements.

\subsection{Circonstances exceptionnelles justifiant le réexamen}

L'article D.612-36-3-1 du Code de l'éducation permet explicitement le réexamen d'une candidature en cas de \textbf{``circonstances exceptionnelles tenant à l'état de santé ou au handicap''}. Mon diagnostic TDAH du 6 septembre 2025, postérieur à la décision de refus, constitue précisément une telle circonstance.

\section{PRÉJUDICES SUBIS ET CARACTÈRE URGENT}

\subsection{Préjudices multiples et graves}

\begin{itemize}
\item \textbf{Préjudice académique} : Interruption d'un parcours d'excellence, impossibilité de poursuivre le projet de thèse avec le Pr. Alain CELISSE (SAMM - Paris 1)
\item \textbf{Préjudice professionnel} : Retard dans mon projet de recherche sur l'early stopping et les réseaux de neurones, impact sur mon projet entrepreneurial NeuroDiv.AI
\item \textbf{Préjudice psychologique} : Sentiment d'injustice, anxiété liée à l'incertitude, impact sur l'estime de soi après des années d'efforts
\item \textbf{Préjudice financier} : Perte d'opportunités de bourses, impossibilité de candidater aux financements doctoraux
\item \textbf{Préjudice administratif} : Complications pour le renouvellement du titre de séjour étudiant
\end{itemize}

\subsection{Urgence de la situation}

L'année universitaire étant en cours, chaque jour de retard aggrave ces préjudices. Une décision rapide permettrait encore mon intégration au second semestre avec les aménagements appropriés.

\section{ÉLÉMENTS ATTESTANT DE MON EXCELLENCE ET POTENTIEL}

\subsection{Parcours académique d'exception malgré le handicap}

\begin{itemize}
\item \textbf{Major de promotion} Master USTHB avec 18.5/20 de moyenne générale
\item \textbf{16/20 en Apprentissage Statistique} à Sorbonne Université (matière fondamentale du MS2A)
\item \textbf{QI de 127-135} attestant de capacités intellectuelles exceptionnelles
\item \textbf{Progression constante} une fois les difficultés identifiées et compensées
\end{itemize}

\subsection{Projet de recherche déjà structuré}

Le Professeur Alain CELISSE (SAMM - Université Paris 1 Panthéon-Sorbonne) a déjà accepté de superviser ma thèse sur le thème : \textbf{``Can Deep Neural Networks fail with non-smooth functions? Early stopping strategies and theoretical guarantees''}. Ce projet :
\begin{itemize}
\item S'inscrit parfaitement dans les axes du M2 MS2A
\item Bénéficie déjà d'un encadrement confirmé
\item Témoigne de ma capacité à mener une recherche de haut niveau
\end{itemize}

\subsection{Innovation et impact sociétal}

Mon projet \textbf{NeuroDiv.AI}, sélectionné par Pépite Sorbonne Université, vise à développer une plateforme d'IA adaptative pour les 150 000 étudiants neuroatypiques en France. Ce projet :
\begin{itemize}
\item Transforme mon expérience personnelle en innovation technologique
\item Répond à un besoin sociétal majeur d'inclusion
\item Démontre ma capacité à allier recherche théorique et applications pratiques
\end{itemize}

\section{DEMANDES FORMULÉES}

Au vu de l'ensemble de ces éléments factuels et juridiques, je sollicite respectueusement :

\subsection{À titre principal}

\begin{enumerate}
\item \textbf{L'annulation de la décision de refus} du 17 juillet 2025 pour erreur manifeste d'appréciation et violation du principe d'égalité
\item \textbf{Le réexamen immédiat de ma candidature} en tenant compte :
   \begin{itemize}
   \item Du diagnostic TDAH et de ses implications
   \item De la correction de l'erreur de notation (14/20 au lieu de 7/20)
   \item De la charge exceptionnelle de 9 modules
   \end{itemize}
\item \textbf{Mon admission au M2 MS2A} avec mise en place d'un Plan d'Accompagnement de l'Étudiant Handicapé (PAEH) incluant :
   \begin{itemize}
   \item Temps supplémentaire (+30\%) pour les examens
   \item Salle d'examen isolée pour limiter les distractions
   \item Possibilité de pauses régulières
   \item Adaptations pédagogiques pour les travaux de groupe
   \end{itemize}
\end{enumerate}

\subsection{À titre subsidiaire}

Conformément au décret n°2021-752, je demande :
\begin{enumerate}
\item La formulation d'au moins \textbf{trois propositions d'admission alternatives} adaptées à mon profil
\item Une \textbf{inscription conditionnelle} au M2 MS2A en auditeur libre avec possibilité de régularisation
\item La mise en place d'une \textbf{commission ad hoc} incluant un spécialiste du TDAH pour réévaluer mon dossier
\end{enumerate}

\subsection{Mesures d'urgence}

\begin{enumerate}
\item Un \textbf{accusé de réception} immédiat de ce recours
\item Un \textbf{entretien} avec les responsables du M2 MS2A sous 15 jours
\item Une \textbf{décision} dans le délai d'un mois conformément aux engagements de la plateforme recoursetudiants.sorbonne-universite.fr
\end{enumerate}

\section{ENGAGEMENT ET GARANTIES}

Je m'engage formellement à :
\begin{itemize}
\item Suivre scrupuleusement les aménagements proposés par le service de santé universitaire
\item Maintenir un suivi médical régulier au Centre Expert Sainte-Anne
\item Collaborer étroitement avec la Mission Handicap pour optimiser mon parcours
\item Partager mon expérience pour améliorer l'inclusion des étudiants neuroatypiques
\end{itemize}

\section{CONCLUSION}

Madame la Présidente, mon parcours démontre qu'avec les adaptations appropriées, je suis capable d'excellence académique. Le TDAH n'est pas un obstacle insurmontable mais une différence cognitive qui, bien accompagnée, peut même devenir source d'innovation et de créativité.

Le refus de ma candidature, basé sur une évaluation biaisée par l'absence d'aménagements et une erreur matérielle de notation, constitue une injustice qui peut et doit être corrigée. Sorbonne Université a l'opportunité de démontrer son engagement pour l'inclusion en accueillant un étudiant déterminé à transformer son handicap en force.

Je reste convaincu que mon admission au M2 MS2A représente un bénéfice mutuel : l'université gagne un étudiant motivé avec un projet de recherche novateur, et je peux enfin exprimer mon plein potentiel dans un environnement adapté.

\vspace{1cm}

\noindent\textbf{Pièces jointes} (9 documents) :
\begin{enumerate}[leftmargin=2cm]
\item Certificat médical détaillé - Diagnostic TDAH (Dr. Boussadia, Centre Expert Sainte-Anne)
\item Bilan neuropsychologique - Tests WAIS-IV (QI 127-135)
\item Copie d'examen corrigée - Module Optimisation (preuve de l'erreur de notation)
\item Relevés de notes USTHB - Major de promotion (18.5/20)
\item Relevés de notes Sorbonne Université - Variations de performances
\item Lettre d'accord - Pr. Alain CELISSE pour supervision de thèse
\item Attestation Pépite Sorbonne Université - Projet NeuroDiv.AI
\item Synthèse juridique - Décret 2021-752 et jurisprudence TDAH
\item Carte étudiante et titre de séjour en cours
\end{enumerate}

\vspace{2cm}

\noindent Je vous prie d'agréer, Madame la Présidente, l'expression de ma très haute considération.

\vspace{2cm}

\begin{flushright}
\begin{minipage}{0.4\textwidth}
\centering
Fait à Paris, le \today\\
\vspace{2cm}
Tarek DJAKER\\
\textit{[Signature]}
\end{minipage}
\end{flushright}

\end{document}