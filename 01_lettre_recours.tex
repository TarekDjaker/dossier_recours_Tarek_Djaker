\documentclass[12pt,french]{scrlttr2}
\usepackage[utf8]{inputenc}
\usepackage[T1]{fontenc}
\usepackage{babel}
\usepackage{graphicx}
\usepackage{geometry}
\usepackage{microtype}
\usepackage{hyperref}

\geometry{a4paper, margin=2.5cm}
\setkomavar{fromname}{<NOM PRÉNOM>}
\setkomavar{fromaddress}{<ADRESSE COMPLÈTE>\\<CODE POSTAL> <VILLE>}
\setkomavar{fromemail}{<EMAIL_ETUDIANT>}
\setkomavar{fromphone}{<TÉLÉPHONE>}
\setkomavar{subject}{Recours administratif – Décision n° <NUM_DECISION> du <JJ/MM/AAAA>}
\setkomavar{signature}{\includegraphics[height=2cm]{signature.png}\\\vspace{0.5cm}<NOM PRÉNOM>\\Étudiant <PARCOURS/ÉTABLISSEMENT>}

\begin{document}

\begin{letter}{
<JURIDICTION/AUTORITÉ>\\
<SERVICE COMPÉTENT>\\
<ADRESSE AUTORITÉ>\\
<CODE POSTAL> <VILLE>
}

\opening{}

\begin{center}
\begin{minipage}{0.3\textwidth}
\centering
\includegraphics[height=1.6cm]{logos/bird_logo.png}
\end{minipage}
\begin{minipage}{0.3\textwidth}
\centering
\includegraphics[height=1.6cm]{logos/pepite_sorbonne.png}
\end{minipage}
\begin{minipage}{0.3\textwidth}
\centering
\includegraphics[height=1.6cm]{logos/certifie_TDAH.webp}
\end{minipage}
\end{center}

\vspace{1cm}

\textbf{Objet :} Recours administratif contre la décision n° <NUM_DECISION> du <JJ/MM/AAAA> – Module « Optimisation » – Rectification de notation

\vspace{0.5cm}

Madame, Monsieur,

Par la présente, je forme un recours administratif contre la décision référencée ci-dessus, relative à l'attribution d'une note erronée au module « Optimisation » de mon parcours <PARCOURS/ÉTABLISSEMENT>.

\section*{I. EXPOSÉ DES FAITS}

Le <DATE_NOTIFICATION>, j'ai constaté sur mon relevé de notes qu'une note de <ANCIENNE NOTE>/20 m'avait été attribuée pour le module « Optimisation » (<CODE_MODULE>). Cette notation ne correspond pas à l'évaluation réelle de ma copie.

Après vérification auprès du responsable pédagogique le <DATE_VERIFICATION>, il a été établi que ma note correcte est de <NOTE CORRECTE>/20, conformément au barème appliqué et aux points obtenus sur ma copie d'examen.

Cette erreur matérielle a des conséquences directes sur :
\begin{itemize}
\item Ma moyenne générale du semestre <SEMESTRE>
\item Mon classement dans la promotion
\item Mes possibilités de poursuite d'études et candidatures
\end{itemize}

Il convient de noter que je suis diagnostiqué TDAH (certificat médical joint) et bénéficie d'aménagements d'examens. Par ailleurs, je suis entrepreneur étudiant accompagné par le dispositif Pépite Sorbonne Université.

\section*{II. RECEVABILITÉ DU RECOURS}

Le présent recours est recevable en la forme :
\begin{itemize}
\item \textbf{Délai :} Formé dans le délai de deux mois suivant la notification de la décision contestée (article R.421-1 du Code de justice administrative)
\item \textbf{Qualité pour agir :} En tant qu'étudiant directement concerné par la décision
\item \textbf{Intérêt à agir :} La décision me fait grief en affectant ma situation académique
\end{itemize}

\section*{III. MOYENS À L'APPUI DU RECOURS}

\subsection*{A. Sur l'erreur matérielle manifeste}

La note attribuée résulte d'une erreur de saisie ou de report, comme l'atteste la discordance entre :
\begin{itemize}
\item Le total des points obtenus sur ma copie : <TOTAL_POINTS> points
\item La note figurant sur le relevé : <ANCIENNE NOTE>/20
\item La note correspondant au barème : <NOTE CORRECTE>/20
\end{itemize}

Cette erreur est objectivement vérifiable par simple consultation de ma copie et du barème de notation.

\subsection*{B. Sur le principe d'égalité de traitement}

L'erreur de notation crée une rupture d'égalité entre étudiants, mes camarades ayant obtenu des notations conformes à leurs performances réelles selon le même barème.

\subsection*{C. Sur l'obligation d'exactitude des actes administratifs}

L'administration universitaire est tenue d'établir des relevés de notes exacts, ces documents ayant valeur officielle et étant utilisés pour :
\begin{itemize}
\item Les candidatures en master ou formations sélectives
\item Les demandes de bourses au mérite
\item Les attestations de réussite
\end{itemize}

\section*{IV. DEMANDES}

Au vu de ce qui précède, je vous demande respectueusement de bien vouloir :

\textbf{À titre principal :}
\begin{enumerate}
\item Annuler ou retirer la décision n° <NUM_DECISION> du <JJ/MM/AAAA> en tant qu'elle m'attribue la note de <ANCIENNE NOTE>/20
\item Établir un nouveau relevé de notes rectifié mentionnant la note correcte de <NOTE CORRECTE>/20
\item Procéder à la mise à jour de mon dossier administratif et numérique
\item Me délivrer une attestation de rectification pour les démarches déjà effectuées
\end{enumerate}

\textbf{À titre subsidiaire :}
\begin{itemize}
\item Organiser une nouvelle correction de ma copie par un jury indépendant
\item Me communiquer l'intégralité des éléments ayant conduit à l'attribution de la note contestée
\end{itemize}

Je reste à votre disposition pour tout complément d'information et vous prie de m'accuser réception du présent recours.

\section*{V. PIÈCES JOINTES}

\begin{enumerate}
\item Relevé de notes erroné du <DATE_RELEVÉ>
\item Relevé de notes corrigé (version provisoire)
\item Copie d'examen avec annotations et barème
\item Certificat médical TDAH du Dr <NOM_MEDECIN>
\item Attestation Pépite Sorbonne Université
\item Échanges e-mail avec le responsable pédagogique
\item Copie carte étudiante n° <NUMERO_ETUDIANT>
\end{enumerate}

Je vous prie d'agréer, Madame, Monsieur, l'expression de ma considération distinguée.

\closing{Fait à <VILLE>, le \today}

\end{letter}
\end{document}