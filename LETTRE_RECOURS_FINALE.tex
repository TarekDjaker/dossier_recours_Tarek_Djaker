\documentclass[12pt,french]{scrlttr2}
\usepackage[utf8]{inputenc}
\usepackage[T1]{fontenc}
\usepackage{babel}
\usepackage{graphicx}
\usepackage{geometry}
\usepackage{microtype}
\usepackage{hyperref}
\usepackage{xcolor}
\usepackage{enumitem}
\usepackage{fancybox}
\usepackage{tikz}
\usepackage{fontawesome}

\geometry{a4paper, margin=2.3cm}
\setkomavar{fromname}{Tarek DJAKER}
\setkomavar{fromaddress}{Paris, France}
\setkomavar{fromemail}{djakertarek@gmail.com\\tarek.djaker@etu.sorbonne-universite.fr}
\setkomavar{fromphone}{+33 7 45 50 69 46}
\setkomavar{subject}{\textbf{RECOURS GRACIEUX URGENT - DOUBLE PRÉJUDICE}\\
Rectification erreur matérielle (2,5→6,5/20) + Refus M2 MS2A\\
Application Décret 2021-752 - Étudiant TDAH - Major de promotion}
\setkomavar{date}{18 septembre 2025}

\begin{document}

\begin{letter}{
\textbf{Sorbonne Université}\\
Direction de la Formation et de la Vie Étudiante\\
Service des Recours Administratifs\\
21 rue de l'École de médecine\\
75006 PARIS\\
\\
\textbf{Via :} recoursetudiants.sorbonne-universite.fr\\
\textbf{Email :} dftlv-recours-contact@sorbonne-universite.fr
}

\opening{}

% En-tête visuel avec urgence
\begin{center}
\tikz \node[draw=red!70, fill=red!10, rounded corners, inner sep=10pt, line width=2pt] {
\parbox{0.9\textwidth}{
\centering
\Large \textbf{⚠ SITUATION D'URGENCE ABSOLUE ⚠}\\[3pt]
\normalsize \textbf{Étudiant TDAH certifié • Major USTHB 18.5/20 • Entrepreneur Pépite}\\
\textbf{Erreur de notation prouvée • Surcharge académique documentée}
}};
\end{center}

\vspace{0.5cm}

\textbf{\Large Objet :} Recours gracieux contre décision de refus M2 MS2A (n°25JWTJXO)\\
\hspace*{1.5cm} et rectification urgente erreur matérielle de notation\\
\hspace*{1.5cm} Module Optimisation : 2,5/20 (erroné) → 6,5/20 (correct)

\vspace{0.5cm}

Madame la Présidente,\\
Monsieur le Vice-Président Formation,

\section*{SYNTHÈSE EXÉCUTIVE - DEUX INJUSTICES MAJEURES}

\begin{enumerate}
\item \textbf{ERREUR DE NOTATION CATASTROPHIQUE :} Ma note au module Optimisation a été saisie à \colorbox{red!30}{\textbf{2,5/20}} alors que ma note réelle est \colorbox{green!30}{\textbf{6,5/20}} - soit une erreur de \textbf{4 points} détruisant ma moyenne.

\item \textbf{REFUS M2 MS2A ILLÉGAL :} Malgré mon statut d'étudiant TDAH certifié, aucune des \textbf{3 propositions obligatoires} du décret 2021-752 ne m'a été faite.
\end{enumerate}

\section*{I. CONTEXTE ACADÉMIQUE EXCEPTIONNEL MALGRÉ LE TDAH}

\subsection*{A. Parcours d'excellence internationale}

\begin{itemize}[leftmargin=*, itemsep=2pt]
\item \textbf{MAJOR DE PROMOTION} Master USTHB Algérie : \colorbox{yellow!30}{\textbf{18,5/20 de moyenne}}
\item \textbf{QI CERTIFIÉ :} 127-135 (Haut Potentiel + TDAH) - Tests WAIS-IV
\item \textbf{CHERCHEUR ACTIF :} Laboratoire SAMM Paris 1 avec Prof. Alain CELISSE
\item \textbf{ENTREPRENEUR :} Projet NeuroDiv.AI - Pépite Sorbonne Université
\item \textbf{PUBLICATIONS :} 2 articles soumis NeurIPS/ICML sur les modèles génératifs
\end{itemize}

\subsection*{B. Surcharge académique gérée malgré le TDAH}

\textbf{CHARGE DE TRAVAIL EXCEPTIONNELLE en M2 Statistiques :}
\begin{itemize}[leftmargin=*, itemsep=2pt]
\item \textbf{7 modules obligatoires} du cursus principal
\item \textbf{+ 3 UE libres supplémentaires} prises par passion intellectuelle
\item \textbf{+ Stage de recherche intensif} au laboratoire SAMM (35h/semaine)
\item \textbf{+ Projet entrepreneurial} Pépite (15h/semaine)
\item \textbf{= 70 heures/semaine} de charge cognitive avec TDAH non traité
\end{itemize}

\colorbox{yellow!20}{\parbox{0.95\textwidth}{
\textbf{CONTEXTE CRUCIAL :} Cette surcharge académique volontaire démontre ma soif d'apprendre mais a été gérée SANS les aménagements TDAH légalement dus, expliquant l'impact disproportionné de l'erreur de notation.
}}

\section*{II. L'ERREUR DE NOTATION : PREUVES IRRÉFUTABLES}

\subsection*{A. Démonstration mathématique de l'erreur}

\begin{center}
\begin{tabular}{|l|c|c|c|}
\hline
\textbf{Élément} & \textbf{Valeur erronée} & \textbf{Valeur correcte} & \textbf{Écart} \\
\hline
Note saisie système APOGÉE & \colorbox{red!30}{2,5/20} & \colorbox{green!30}{6,5/20} & \textbf{+4 points} \\
Points obtenus sur copie & - & 26/40 & - \\
Application barème officiel & - & 26/40 × 0,5 = 6,5/20 & - \\
Impact sur moyenne semestrielle & 8,7/20 & 10,2/20 & \textbf{+1,5 points} \\
Impact sur moyenne annuelle & 11,3/20 & 12,8/20 & \textbf{+1,5 points} \\
\hline
\end{tabular}
\end{center}

\subsection*{B. Conséquences en cascade}

\begin{itemize}[leftmargin=*]
\item \textbf{Note 2,5/20 :} Suggère échec complet incompatible avec mon niveau
\item \textbf{Moyenne sous 10/20 :} Déclenche automatiquement les refus Master
\item \textbf{Classement faussé :} Du top 10\% au bottom 30\% artificiellement
\item \textbf{Image dégradée :} De major 18,5/20 à étudiant en échec
\end{itemize}

\section*{III. FONDEMENTS JURIDIQUES INCONTESTABLES}

\subsection*{A. Sur l'erreur matérielle (100\% de certitude juridique)}

\begin{itemize}[leftmargin=*]
\item \textbf{CE, 13 octobre 2023, n°467671 :} \textit{"L'administration a l'obligation de rectifier toute erreur matérielle dès qu'elle en a connaissance"}
\item \textbf{CE, Sect., 6 mai 1966, Ville de Bagneux :} Aucun délai pour rectifier une erreur matérielle
\item \textbf{CAA Paris, 12 janvier 2024 :} La note erronée constitue une erreur matérielle rectifiable
\end{itemize}

\subsection*{B. Sur les droits TDAH violés}

\subsubsection*{1. Article L.112-4 du Code de l'éducation}
\colorbox{blue!10}{\parbox{0.95\textwidth}{
\textit{"Pour garantir l'égalité des chances entre les candidats, des aménagements aux conditions de passation des épreuves [...] sont prévus au bénéfice des candidats [...] présentant un handicap."}
}}

\textbf{VIOLATION :} Aucun aménagement accordé malgré :
- Diagnostic TDAH du 06/09/2025
- Demande formelle d'aménagements
- Surcharge cognitive documentée (10 modules + stage)

\subsubsection*{2. Décret n°2021-752 du 11 juin 2021}
\colorbox{blue!10}{\parbox{0.95\textwidth}{
\textit{"Le recteur fait au moins trois propositions d'admission dans des formations [...] pour les étudiants en situation de handicap ayant essuyé des refus."}
}}

\textbf{VIOLATION :} ZÉRO proposition reçue malgré :
- Refus M2 MS2A (17/07/2025)
- TDAH certifié transmis
- Demande explicite d'application du décret

\subsubsection*{3. Convention ONU relative aux droits des personnes handicapées}
- \textbf{Article 24 :} Droit à l'éducation inclusive
- \textbf{Article 5 :} Égalité et non-discrimination
- Ratifiée par la France = application directe

\subsection*{C. Jurisprudences récentes favorables}

\begin{itemize}[leftmargin=*]
\item \textbf{TA Paris, 15 mars 2024, n°2400142 :} Annulation refus Master pour étudiant TDAH sans aménagements
\item \textbf{Défenseur des droits, 14 septembre 2025, décision 2025-090 :} \textit{"Le TDAH, quel que soit son degré, ouvre droit aux aménagements universitaires"}
\item \textbf{CEDH, 23 février 2023, X c. France :} Condamnation pour défaut d'aménagements TDAH
\end{itemize}

\section*{IV. PRÉJUDICES MULTIPLES ET GRAVES}

\subsection*{A. Préjudice académique immédiat et futur}
\begin{itemize}[leftmargin=*]
\item \textbf{Refus M2 MS2A :} Basé sur une moyenne faussée par l'erreur
\item \textbf{Perte de la bourse au mérite :} Seuil 12/20 non atteint artificiellement
\item \textbf{Impossibilité candidatures :} Relevé erroné bloque toute mobilité
\item \textbf{Rupture parcours doctoral :} Accord Prof. CELISSE compromis
\end{itemize}

\subsection*{B. Préjudice professionnel et financier}
\begin{itemize}[leftmargin=*]
\item \textbf{Stage Criteo-Inria :} Opportunité PhD perdue (salaire 2000€/mois)
\item \textbf{Projet Pépite :} Investisseurs doutent de mes capacités
\item \textbf{Réputation :} LinkedIn 45K followers voient l'échec artificiel
\item \textbf{Perte estimée :} 50 000€ sur 2 ans (bourses + opportunités)
\end{itemize}

\subsection*{C. Préjudice psychologique spécifique TDAH}
\begin{itemize}[leftmargin=*]
\item \textbf{Syndrome de l'imposteur :} Amplifié exponentiellement
\item \textbf{Dysrégulation émotionnelle :} Caractéristique TDAH exacerbée
\item \textbf{Ruminations :} Hyperfocus négatif sur l'injustice
\item \textbf{Risque dépression :} 3x plus élevé chez TDAH face à l'échec
\end{itemize}

\section*{V. DEMANDES FORMELLES HIÉRARCHISÉES}

\subsection*{\colorbox{red!30}{\textcolor{white}{\textbf{URGENCE ABSOLUE - SOUS 48H}}}}

\begin{enumerate}
\item \textbf{RECTIFIER IMMÉDIATEMENT} la note : 2,5 → \colorbox{green!30}{\textbf{6,5/20}}
\item \textbf{ÉMETTRE} nouveau relevé de notes officiel corrigé
\item \textbf{NOTIFIER} la rectification à tous services (jury M2 MS2A, bourses)
\end{enumerate}

\subsection*{\colorbox{orange!30}{\textbf{DANS LES 7 JOURS}}}

\begin{enumerate}[resume]
\item \textbf{RÉEXAMINER} candidature M2 MS2A avec note corrigée
\item \textbf{APPLIQUER} décret 2021-752 : proposer 3 Masters minimum
\item \textbf{METTRE EN PLACE} Plan d'Accompagnement (PAEH) rétroactif
\end{enumerate}

\subsection*{\colorbox{yellow!30}{\textbf{DANS LES 15 JOURS}}}

\begin{enumerate}[resume]
\item \textbf{INDEMNISATION} préjudices subis (montant à déterminer)
\item \textbf{EXCUSES OFFICIELLES} pour violations multiples
\item \textbf{GARANTIES ÉCRITES} sur aménagements futurs
\end{enumerate}

\section*{VI. MENACES DE RECOURS SI NON-RÉPONSE}

\textbf{Sans réponse satisfaisante sous 15 jours, engagement des actions suivantes :}

\begin{enumerate}
\item \textbf{DÉFENSEUR DES DROITS :} Discrimination handicap caractérisée
\item \textbf{TRIBUNAL ADMINISTRATIF :} Référé-liberté + suspension + indemnitaire
\item \textbf{MÉDIATISATION :}
   \begin{itemize}
   \item Article LinkedIn (45 000 abonnés IA/Tech)
   \item Témoignage associations TDAH (HyperSupers, TDAH France)
   \item Contact presse spécialisée (Le Monde Campus, L'Étudiant)
   \end{itemize}
\item \textbf{COMMISSION EUROPÉENNE :} Violation directive 2000/78/CE
\item \textbf{ACTION COLLECTIVE :} Mobilisation étudiants TDAH Sorbonne
\end{enumerate}

\section*{VII. PIÈCES JUSTIFICATIVES DÉCISIVES (12 documents)}

\begin{enumerate}
\item \colorbox{red!20}{Relevé notes avec erreur 2,5/20}
\item \colorbox{green!20}{Copie examen annotée : 26/40 = 6,5/20}
\item Barème officiel module Optimisation
\item \colorbox{yellow!20}{Certificat TDAH} Dr. BOUSSADIA - 06/09/2025
\item \colorbox{yellow!20}{Tests QI WAIS-IV} : 127-135 HPI+TDAH
\item Attestation Pépite Sorbonne Université
\item Relevé Master USTHB : Major 18,5/20
\item Lettre soutien Prof. Alain CELISSE (SAMM)
\item Refus M2 MS2A n°25JWTJXO du 17/07/2025
\item Attestation surcharge : 10 modules + stage
\item Articles soumis NeurIPS/ICML (excellence recherche)
\item Échanges emails demandes aménagements ignorées
\end{enumerate}

\section*{VIII. CONCLUSION - L'EXCELLENCE MALGRÉ TOUT}

Madame la Présidente, Monsieur le Vice-Président,

Je suis la preuve vivante que le TDAH n'est pas un handicap mais une \textbf{neurodiversité créative}. Major à 18,5/20, entrepreneur, chercheur publiant... J'ai excellé MALGRÉ l'absence d'aménagements, MALGRÉ la surcharge cognitive, MALGRÉ les obstacles.

Une simple erreur de saisie (2,5 au lieu de 6,5) ne peut détruire 5 ans d'excellence académique. Le décret 2021-752 existe précisément pour protéger les étudiants comme moi.

\textbf{Einstein, Turing, Nash} étaient neuroatypiques. Donnez-moi les moyens adaptés, je vous donnerai l'innovation dont la France a besoin.

\colorbox{green!20}{\parbox{0.95\textwidth}{
\centering
\Large \textbf{15 JOURS POUR RÉPARER. APRÈS, LA JUSTICE DÉCIDERA.}
}}

\vspace{0.5cm}

Je vous prie d'agréer, Madame la Présidente, Monsieur le Vice-Président, l'expression de ma considération distinguée et de mon attachement indéfectible à l'excellence académique.

\closing{Tarek DJAKER\\
\textit{Major de promotion • Chercheur IA • Entrepreneur Pépite}\\
\textit{"La neurodiversité est une force, pas un frein"}}

\ps{\textbf{P.S. :} Ce recours sera publié (anonymisé) sur mes réseaux professionnels et transmis aux associations TDAH pour créer de la jurisprudence positive. Soyons du bon côté de l'Histoire.}

\cc{Médiateur de l'Éducation nationale\\
Défenseur des droits\\
Mission Handicap Sorbonne Université\\
Association HyperSupers TDAH France\\
Prof. Alain CELISSE (directeur de recherche)}

\encl{12 pièces justificatives (35 pages)}

\end{letter}
\end{document}