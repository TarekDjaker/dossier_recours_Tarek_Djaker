\documentclass[11pt,a4paper]{article}
\usepackage[T1]{fontenc}\usepackage[utf8]{inputenc}\usepackage[french]{babel}
\usepackage{geometry}\geometry{margin=2.5cm}
\usepackage{lmodern}\usepackage{hyperref}\hypersetup{colorlinks=true, linkcolor=black, urlcolor=blue}

\begin{document}
\begin{flushright}
\textbf{Tarek DJAKER}\\
<Adresse>\\
\href{mailto:djakertarek@gmail.com}{djakertarek@gmail.com} \;–\; +33\,7\,45\,50\,69\,46\\
\end{flushright}

\vspace{0.5cm}
\textbf{À l’attention de :} Madame la Présidente de Sorbonne Université\\
\textbf{Copie :} Vice‑Présidence Formation ; Direction de la scolarité ; Service handicap (SHSE)

\vspace{0.5cm}
\textbf{Objet :} \textit{Recours gracieux – Rectification d’une erreur matérielle de notation et réexamen de mon admission en M2 MS2A}

\vspace{0.3cm}
\textbf{Références :}
\begin{itemize}
  \item Décision contestée : refus d’admission en M2 MS2A du <date/numéro>
  \item Module « Optimisation » – note saisie 2,5/20 au lieu de 6,5/20 (26/40 au barème)
\end{itemize}

Madame, Monsieur,

\textbf{1. Faits et rectification sollicitée.}
À l’issue de l’année 2024–2025, une \textit{erreur matérielle de notation} a été commise pour le module « Optimisation » :
ma copie comporte 26/40 points, correspondant à \textbf{6,5/20} selon le barème communiqué, alors que \textbf{2,5/20} a été reporté
sur mon relevé. Je sollicite la \textbf{rectification} de cette erreur et, le cas échéant, une \textbf{nouvelle délibération} afin d’en tirer
toutes les conséquences académiques.

\textbf{2. Situation actuelle et projet académique.}
Je suis également \textbf{inscrit en M2 « Mathématiques appliquées à la Data Science » (Université Sorbonne Paris‑Nord)} afin
d’assurer la continuité de mon parcours. Par ailleurs, j’ai échangé avec le \textbf{Pr. Franck Iutzeler} au sujet d’une \textbf{perspective
de thèse en 2026}, sous réserve de financement – une demande ayant été déposée en juillet dernier.
Mon \textbf{projet scientifique} (optimisation, apprentissage statistique) est pleinement \textbf{cohérent} avec le M2 MS2A de Sorbonne
Université, qui demeure mon vœu principal.

\textbf{3. Aménagements et information liés au TDAH.}
Un diagnostic \textbf{TDAH} a été posé récemment. N’ayant pas identifié à temps les dispositifs
d’accompagnement (aménagements d’épreuves, information dédiée), je n’ai pas sollicité les aides
adaptées durant l’année écoulée. Je regrette ce \textit{défaut d’information} de ma part et souhaite désormais
formaliser un \textbf{PAEH} avec le service compétent, afin d’étudier des aménagements raisonnables.
Je reste bien entendu disponible pour fournir les pièces médicales utiles.

\textbf{4. Demandes.}
\begin{itemize}
  \item (\textbf{Rectification}) Corriger la note du module « Optimisation » à \textbf{6,5/20} et mettre à jour mon relevé ;
  \item (\textbf{Réexamen}) Procéder au \textbf{réexamen} de mon admission en M2 MS2A à la lumière de cette rectification ;
  \item (\textbf{Accompagnement}) Organiser un rendez‑vous avec le \textbf{SHSE/Mission Handicap} pour mettre en place un PAEH ;
  \item (\textbf{Communication}) Me transmettre, au titre du CRPA, le \textbf{barème détaillé}, le \textbf{PV de jury} et les \textbf{critères d’admission}
        utilisés pour le M2.
\end{itemize}

\textbf{5. Esprit constructif.}
Je m’engage, si vous en convenez, à \textbf{participer bénévolement} aux actions de sensibilisation à l’attention
des étudiants concernés (PÉPITE, ateliers d’information handicap) afin d’améliorer la diffusion des dispositifs
d’accompagnement et d’éviter qu’une situation analogue ne se reproduise.

Je vous remercie par avance de l’attention portée à ce recours. Je demeure à votre disposition pour tout échange
utile et pour fournir toute pièce complémentaire.

Veuillez agréer, Madame, Monsieur, l’expression de ma considération distinguée.

\vspace{0.3cm}
\textbf{Tarek DJAKER}

\textit{Pièces jointes (extraits)} : copie d’examen (26/40), barème, relevé portant 2,5/20, certificat médical TDAH, attestations de parcours, preuves d’inscription M2 USPN, attestation d’échange scientifique (Pr. Iutzeler), etc.
\end{document}
