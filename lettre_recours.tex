\
\documentclass[11pt]{scrlttr2}
\usepackage[T1]{fontenc}
\usepackage[utf8]{inputenc}
\usepackage[french]{babel}
\usepackage{geometry}
\geometry{margin=2.5cm}
\usepackage{graphicx,xcolor}
\setkomavar{fromname}{<NOM PRÉNOM>}
\setkomavar{fromaddress}{<ADRESSE COMPLÈTE>}
\setkomavar{fromemail}{<EMAIL>}
\setkomavar{fromphone}{<TÉLÉPHONE>}
\setkomavar{signature}{<NOM PRÉNOM>}
\setkomavar{subject}{Recours <gracieux/hiérarchique/contentieux> contre la décision n° <NUM_DECISION> du <JJ/MM/AAAA>}

\firsthead{
  \noindent
  \begin{minipage}[t]{0.32\linewidth}
    \includegraphics[height=1.6cm]{logos/bird_logo.png}
  \end{minipage}\hfill
  \begin{minipage}[t]{0.32\linewidth}
    \centering
    \includegraphics[height=1.6cm]{logos/pepite_sorbonne.png}
  \end{minipage}\hfill
  \begin{minipage}[t]{0.32\linewidth}
    \raggedleft
    \includegraphics[height=1.6cm]{logos/certifie_TDAH.webp}
  \end{minipage}
  \par\vspace{0.3em}\hrule
}

\begin{document}
\begin{letter}{<JURIDICTION / AUTORITÉ>\\<ADRESSE OFFICIELLE>\\<EMAIL OFFICIEL (si envoi électronique autorisé)>}
\opening{Madame, Monsieur,}

\textbf{Références :} Décision n° \textbf{<NUM_DECISION>} rendue le \textbf{<JJ/MM/AAAA>} par \textbf{<AUTORITÉ>}.\\
\textbf{Objet :} Recours \textbf{<gracieux/hiérarchique/contentieux>} — \textit{demande d'annulation et de réexamen}.

\vspace{0.6em}
\textbf{1. Faits et éléments objectifs.}
Je suis inscrit en <PARCOURS / ÉTABLISSEMENT>. La note attribuée au module \textbf{Optimisation} a été calculée/communiquée de manière erronée : une \textbf{relevée de notes} récente fait apparaître que ma note correcte est \textbf{<NOTE CORRECTE>} (et non \textbf{<ANCIENNE NOTE / 2.5>}), ce qui impacte significativement ma moyenne et mes perspectives académiques et professionnelles.\\
Par ailleurs, je suis engagé dans un \textbf{projet d'entrepreneuriat} (\textit{Pépite Sorbonne Université}) et dispose d'un \textbf{diagnostic TDAH} attesté (\textit{certificat joint}), facteurs devant être pris en compte au titre de l'\textit{aménagement des études/épreuves}.

\vspace{0.5em}
\textbf{2. Recevabilité et délai.}
Le présent recours est formé dans le délai légal applicable à compter de la notification intervenue le \textbf{<DATE NOTIFICATION>}.

\vspace{0.5em}
\textbf{3. Moyens (fondements).}
\begin{itemize}
  \item \textbf{Erreur matérielle de notation / de transcription} : discordance entre la copie de correction, le barème, et le relevé officiel (Pièces P2–P4).
  \item \textbf{Vice de procédure / information} : absence d'information claire sur la méthode de calcul et voies de recours ; défaut de communication des éléments ayant servi à la note (CRPA L311-1 s.).
  \item \textbf{Prise en compte des aménagements} : au regard de mon \textbf{TDAH certifié} (Pièce P5), j'aurais dû bénéficier d'aménagements (temps majoré, conditions matérielles), ce qui affecte l'appréciation de la performance.
  \item \textbf{Proportionnalité et intérêt de l'étudiant} : conséquences manifestement excessives au regard de mon parcours et de l'intérêt du service ; demande de réexamen loyal.
\end{itemize}

\vspace{0.5em}
\textbf{4. Demandes précises.}
\begin{itemize}
  \item \textbf{Annuler} la décision litigieuse de validation de note telle que communiquée ;
  \item \textbf{Rectifier} officiellement la note du module \textbf{Optimisation} à \textbf{<NOTE CORRECTE>} \textit{ou}, à titre subsidiaire, \textbf{organiser une épreuve de substitution} avec aménagements adaptés ;
  \item \textbf{Mettre à jour} mon \textbf{relevé de notes} et attestation de réussite ; 
  \item \textbf{Notifier} la décision motivée sous \textbf{<DÉLAI (ex. 15 jours ouvrés)>}.
\end{itemize}

\vspace{0.5em}
\textbf{5. Pièces jointes (bordereau).}
P1—Lettre de recours ; P2—Relevé de notes erroné ; P3—Relevé corrigé / échanges ; P4—Barème/copie corrigée ; P5—Certificat TDAH ; P6—Attestation Pépite ; P7—Tout document utile.

\vspace{0.5em}
Je reste à disposition pour tout échange et, le cas échéant, une médiation. Je vous prie d’agréer, Madame, Monsieur, l'expression de ma considération distinguée.

\closing{<NOM PRÉNOM>}
\end{letter}
\end{document}
